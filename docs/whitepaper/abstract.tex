This paper introduces a novel platform where developers can find virtual or physical servers suitable for their requirements, offered by non-anonymous node providers, and backed by strong reputation checks and legal contracts and guarantees. The servers are offered by 'node providers' which, fundamentally different from typical decentralized cloud platforms, have to identify and build their reputation over long time. This allows users to pick node providers with high reputations if they wish so.

This approach addresses multiple challenges inherent to conventional centralized cloud computing, such as vendor lock-in, high prices, and difficult access to large user bases and thus difficult growth of new node providers. Additionally, it overcomes certain limitations of existing decentralized cloud computing platforms by offering high-performance servers and a familiar development environment. The reputation tracking system is the first decentralized cloud platform that provides sufficient confidence into the node provider.

The system supports sensitive data processing and storage using Confidential Computing VMs, and also facilitates the rental of GPU nodes for Machine Learning (ML) and Artificial Intelligence (AI) training and inference applications at reasonable market prices, catering to different developer preferences. Some developers may opt for the lowest cost GPUs irrespective of node provider reputation and confidentiality guarantees, while others may prefer high-end GPUs with higher node provider reputation. This flexibility accommodates all developer categories, a feature unique to this platform.

The platform runs the control plane on a blockchain, where financial transactions and reputation are tracked and orchestrated. The performance-sensitive data plane runs outside of a blockchain, on the regular Internet. This approach eliminates the biggest issue of conventional web3 platforms. With Decent Cloud developers do not have to reengineer and rewrite legacy applications, while they still have the guarantees of web3.
